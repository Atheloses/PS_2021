\documentclass{article}% use option titlepage to get the title on a page of its own.
\usepackage{booktabs}
\usepackage{indentfirst}
\usepackage{listings}
\usepackage{hyperref}
\usepackage[autostyle=true,czech=quotes]{csquotes}

\renewcommand \thesection {\arabic{section}}
\renewcommand \thesubsection {\thesection.\arabic{subsection}}
\renewcommand \thesubsubsection {\thesubsection.\Alph{subsubsection}}

\title{Pravděpodobnost a statistika\\
    \large Domácí úkoly}

\date{28.3.2021}
\author{Martin Pustka} 

\begin{document}

\maketitle
\newpage
\tableofcontents
\newpage
\section{1P}
    Děláno we wordu :(
\newpage
\section{2P}

\subsection{}
Za poslední tři roky firma uzavřela 9 projektů se státní dotací. K těmto projektům vyhotovila závěrečné zprávy, přičemž v závěrečné dokumentaci ke 3 projektům jsou závažné chyby. Auditor si z 9 projektů  vybral 4 ke kontrole, náhodně. Jestliže X vyjadřuje počet projektů bez chyb v dokumentaci mezi projekty vybranými ke kontrole určete: 

\subsubsection{}
Pravděpodobnostní a distribuční funkci této náhodné veličiny X. Pravděpodobnostní funkci zapište do tabulky a distribuční funkci zadejte předpisem. 
\begin{lstlisting}[language=R, showstringspaces=false, basicstyle=\small]
    # Pouzijeme Hypergeometrickou vetu, vybirame pocet uspechu
    # X ... pocet projektu bez chyb mezi 4 vybranymi
    # X ~ H(N = 9, M = 6, n = 4)

    x = 0:4 #pocet projektu, ktere nas zajimaji
    N = 9 #celkem 9 projektu
    M = 6 #celkovy pocet projektu, ktere nas zajimaji
    n = 4 #pocet projektu ve vyberu

    # graf pravdepodobnostni funkce pro P(X = 0 az 4)
    p = dhyper(x, M, N - M, n) # hodnoty pravd. funkce pro x
    #kontrola - melo by dat 1
    #sum(p) == 1
    #plot(x, p)

    #jednotlive pravdepodobnosti p priradime k odpovidajicim 
    #hodnotam vytyhnutemu poctu spravnych projektu
    pravdep = rbind(c("P(xi)",p))
    colnames(pravdep) = c("xi",0:4)
    rownames(pravdep) = ""
    pravdep #tabulka pravdepodobnostni funkce
    pravd.f(x, p) #vykresleni

    #pravdepodobnosti prubezne scitame a priradime k nim omezeni
    distrib = cbind(cumsum(p),c("x<=0","0<x<=1","0<x<=1","1<x<=2","x>3"))
    colnames(distrib) = c("","")
    rownames(distrib) = c("","","F(x) = ","","")
    distrib #predpis distribucni funkce
    dist.f(x,p) #vykresleni
\end{lstlisting}

\subsubsection{}
Střední hodnotu, rozptyl, směrodatnou odchylku a modus náhodné veličiny X. 
\begin{lstlisting}[language=R, showstringspaces=false, basicstyle=\small]
    # stredni hodnota je soucet vsech hodnot vynasobenych jejich pravdepodobnosti
    # rozptyl pomoci E(x^2) - E(x)^2
    # smerodatna odchylka je odmocnina rozptylu
    # modus je hodnota nejvetsi pravdepodobnosti
    # vypocet modus by bylo treba nahradit y[matchAll(max(p),p)] 
    # misto y[match(max(p),p)], ale chtel jsem to mit spustitelne 
    # bez knihoven (matchAll je asi v knihovne tuple?)
    souhrn(x,p)
\end{lstlisting}

\subsubsection{}
Jaká je pravděpodobnost, že mezi projekty vybranými ke kontrole bude alespoň jeden projekt uzavřený s chybami v závěrečné dokumentaci? 
\begin{lstlisting}[language=R, showstringspaces=false, basicstyle=\small]
    # vybirame alespon jeden spatny projekt, 
    # takze 1 nebo 2 nebo 3 spravne (4 spatne neexistuji)
    # P(x<=3)
    sum(p[1:4])
\end{lstlisting}

\subsubsection{}
Je stanoveno, že pokud auditor shledá v projektu závažná pochybení, firma dostane za každý projekt uzavřený s chybami pokutu 200 000 Kč.
Určete očekávanou výši pokuty, kterou by měla firma zaplatit. 
\begin{lstlisting}[language=R, showstringspaces=false, basicstyle=\small]
    # celkem 4 projekty a odecteme sanci na vytazeni spravnych projektu
    (4 - sum(x*p)) * 200000 
\end{lstlisting}

\subsubsection{}
S jakou pravděpodobností bude pokuta větší než 250 000 Kč? 
\begin{lstlisting}[language=R, showstringspaces=false, basicstyle=\small]
    # vetsi nez 250 tisic bude v pripade, ze vybereme alespon dva spatne projekty
    # P(x<3)
    sum(p[1:3])
\end{lstlisting}

\newpage
\subsection{}
Náhodná veličina Y je dána distribuční funkcí

\subsubsection{}
Načrtněte graf distribuční i pravděpodobnostní funkce náhodné veličiny Y. 
\begin{lstlisting}[language=R, showstringspaces=false, basicstyle=\small]
    # definujeme distribucni funkci
    F = c(0, 0.3, 0.5, 0.9, 1)
    y = c(1,10,100,1000)
    
    # z distribucni funkce zjistime pravdepodobnosti pro omezeni
    p = diff(F)
    pravd.f(y, p) #vykreslime pravdepodobnosti funkci
    dist.f(y, p) #vykreslime distribucni funkci    
\end{lstlisting}

\subsubsection{}
Pomocí distribuční funkce vyjádřete a vypočtěte pravděpodobnost, že Y=100, a že hodnota Y je alespoň 10. 
\begin{lstlisting}[language=R, showstringspaces=false, basicstyle=\small]
    # P(Y=100) 
    p[match(100,y)] #dohledame hodnotu pro 100
    # P(Y >= 10)
    #hledame vetsi nez, takze musime odecist pravdepodobnost 10
    1-F[match(10,y)] 
    # nebo by jsme mohli secist zmeny vetsich nez    
    p[match(10,y)] + p[match(100,y)] + p[match(1000,y)] 
\end{lstlisting}

\subsubsection{}
Určete střední hodnotu, rozptyl, směrodatnou odchylku a modus náhodné veličiny Y. 
\begin{lstlisting}[language=R, showstringspaces=false, basicstyle=\small]
    souhrn(y, p) # podobne jako 1B
\end{lstlisting}

\subsubsection{}
Jestliže pro náhodnou veličinu R platí R=log(Y), určete střední hodnotu a rozptyl náhodné veličiny R. 
\begin{lstlisting}[language=R, showstringspaces=false, basicstyle=\small]
    R = log10(y) # prepocteme si omezeni podle logaritmu

    # mohlo by dojit k nesetrizenym omezenim, takze je treba setridit
    idx_sorted = order(R)
    R = R[idx_sorted]
    p_R = p[idx_sorted]
    
    souhrn(R, p_R)    
\end{lstlisting}

\newpage
\subsection{}
Rozdělení pravděpodobnosti spojité náhodné veličiny (SNV) X je dáno hustotou 

\subsubsection{}
Určete hodnotu konstanty c. Načrtněte graf hustoty f(x). 
\begin{lstlisting}[language=R, showstringspaces=false, basicstyle=\small]
    # integral od -2/3 do 0 pro f(x)dx = 1 kde f(x) = c(6x+8)
    # c(3*x^2+8*x) => c(0*0)-c(3*(-2/3)^2+8*(-2/3)) => 
    # => c(0) - c(-4) => c*4 dame rovno jedne => c = 0.25
    f = function(x){return(6*x+8)} # f(x) = 6x+8
    dolniMez = -2/3
    horniMez = 0
    1/integrate(f, dolniMez, horniMez)$value
    
    # na papire si vyberu dva krajni body -2/3 a 0 a vypocitam jejich hodnoty
    # jelikoz jde o primku tak tyto dva body bych spojil a uzavreny interval
    # zbytek grafu hustoty bude na nule do nekonecen s otevrenymi intervaly
    
    f.dens = function(x){
        res = 6/4*x+2 # 1/4(6*x+8)
        res[x < -2/3] = 0 
        res[x > 0] = 0  
        return(res)
    }
    
    x = seq(from = -1, to = 0.5, by = 0.01) 
    FX = f.dens(x) 
    plot(x, FX, cex = 0.2, main="Graf hustoty")     
\end{lstlisting}

\subsubsection{}
Určete distribuční funkci F(x). 
\begin{lstlisting}[language=R, showstringspaces=false, basicstyle=\small]
    # integral od -2/3 do 0 pro F(t) = f(x)dx
    #       0                   t < -2/3
    # F(t)  6/8*t^2 + 2*t + 1   t <-2/3;0>
    #       1                   t > 0
    # pocitam integral od -2/3 do t, kde t je mensi nez 0, pro f(x)dx
    # 6/4*x+2
    # 6/8*x^2+2*x
    # 6/8*t^2+2*t - (6/8*(-2/3)^2+2*(-2/3)) => 6/8*t^2 + 2*t + 1 => dosadim do F(t)
    
    F.dist = function(x){
        res = 6/8*x^2 + 2*x + 1
        res[x < -2/3] = 0 
        res[x > 0] = 1  
        return(res)
    }
    
    x = seq(from = -1, to = 0.5, by = 0.01) 
    FX = F.dist(x)
    plot(x, FX, type = 'l', main="Distribucni funkce")
\end{lstlisting}

\subsubsection{}
Vypočtěte pravděpodobnosti P(-1<=X<=-1/3) a P(X>-1/3). 
\begin{lstlisting}[language=R, showstringspaces=false, basicstyle=\small]
    # vypocet se provadi integraci hustoty pravdepodobnosti v limitach danych zadanim,
    # pripadne oseknutou limitami hustoty integrace provedena v 3B, 
    # pouze dosazujeme (6/8*(-1/3)^2 + 2*(-1/3) + 1) - (6/8*(-2/3)^2 + 2*(-2/3) + 1)
    # P(-1 <= X <= -1/3)
    integrate(f.dens, -2/3, -1/3)$value
    
    # P(X > -1/3)
    # dosazujeme (6/8*(0)^2 + 2*(0) + 1) - (6/8*(-1/3)^2 + 2*(-1/3) + 1)
    integrate(f.dens, -1/3, 0)$value 
\end{lstlisting}

\subsubsection{}
Vypočtěte číselné charakteristiky SNV X. (E(X)=?, D(X)=?, sigma.X=?) 
\begin{lstlisting}[language=R, showstringspaces=false, basicstyle=\small]
    x_fx = function(x){
        fx = f.dens(x)
        return(x*fx)
    } 
    xx_fx = function(x){
        fx = f.dens(x) 
        return(x*x*fx)
    } 
    
    # pouzijeme vzorce E(x) = integral x*f(x)dx na celem nenulovem intervalu
    # E(x^2) = integral x^2*f(x)dx, pro D(x) = E(x^2) - E(x)^2
    E_X = integrate(x_fx, -2/3, 0)$value
    E_XX = integrate(xx_fx, -2/3, 0)$value
    
    D_X = E_XX - E_X^2
    std_X = sqrt(D_X)
    
    cbind(c("E(X)","D(X)","sigma x"),c(E_X,D_X,std_X))
\end{lstlisting}

\subsubsection{}
Pro SNV Y platí, že Y=4-2X. Určete střední hodnotu E(Y) a rozptyl D(Y). 
\begin{lstlisting}[language=R, showstringspaces=false, basicstyle=\small]
    # pro Y = 4-2x vyuzijeme vzorecku
    # E(aX + b) = a*E(x)+b
    # D(aX + b) = a^2*D(x)
    # a hodnot, ktere jsme vypocitali v 3D
    E_Y = 4 - 2*E_X # E(4-2x)
    D_Y = (-2)^2*D_X # D(4-2x)
    std_Y = sqrt(D_Y)
    
    cbind(c("E(Y)","D(Y)","sigma y"),c(E_Y,D_Y,std_Y))
\end{lstlisting}


\newpage
\subsection{}
Spojitá náhodná veličina X je popsána distribuční funkcí 

\subsubsection{}
Určete hustotu pravděpodobnosti f(x). 
\begin{lstlisting}[language=R, showstringspaces=false, basicstyle=\small]
    # hustota se pocita derivaci distribucni funkce
    # zderivujeme sin x => cos x
    # f(x)  cos x <0,pi/2>
    #       0 (-inf,0) nebo (pi/2,inf)
    f = function(x){
        res = cos(x)     # x^2+2x+1
        res[x < 0] = 0 # 0 pro x<=0
        res[x > pi/2] = 0  # 1 pro x>1
        return(res)
    }
    
    x = seq(from = -0.5, to = 2, by = 0.01) 
    FX = f(x) 
    plot(x, FX, cex = 0.2, main="Graf hustoty")     
\end{lstlisting}

\subsubsection{}
Určete medián x0,5. 
\begin{lstlisting}[language=R, showstringspaces=false, basicstyle=\small]
    # median nalezi do <0,pi/2> a pocita se z distribucni funkce
    F.dist = function(x){
        res = sin(x)     # x^2+2x+1
        res[x <= 0] = 0 # 0 pro x<=0
        res[x > pi/2] = 1  # 1 pro x>1
        return(res)
    }
\end{lstlisting}

\subsubsection{}
Najděte w takové, aby pravděpodobnost, že hodnota SNV X bude větší než w, byla 60 \%. 
\begin{lstlisting}[language=R, showstringspaces=false, basicstyle=\small]
    # podobne jako median spocitame i pravdepodobnost 60%, 
    # akorat musime pocitat s 1-0.6
    x = seq(from = -0.5, to = 2, by = 0.001) 
    FX = F.dist(x) 
    plot(x, FX, type='l', main="Pravdepodobnostni funkce a pravdepodobnost 60%") 
    lines(c(-0.5, 2),c(0.4, 0.4))
    
    asin(0.4)
    x[FX >= 0.4][1]
\end{lstlisting}


\newpage
\section{3P}

\subsection{}
V přírodní rezervaci se v průměru nachází 1 stádo divoké zvěře na 25 $km^2$. Celá rezervace je, co se týče vegetace, zdrojů potravy či vody, vyvážená, tudíž neočekáváme, že by v některých oblastech docházelo k nerovnoměrné kulminaci více stád.

\subsubsection{}
a) Jaká je pravděpodobnost, že ve zkoumané oblasti rezervace o rozloze 100 $km^2$ bude nejvýše 6 stád?

\begin{lstlisting}[language=R, showstringspaces=false, basicstyle=\small]
    # X... pocet stad na 1km^2
    # X - pouzijeme Poissonovu vetu
    lambda = 1/25
    t = 100
    lt = lambda*t # parametr Poissonova rozdeleni
    
    # P(x <= 6)
    # V oblasti 100km^2 bude nejvyse 6 stad
    ppois(6, lt) # s pravdepodobnosti 88,93 %    
\end{lstlisting}

\subsubsection{}
b) Jaká je pravděpodobnost, že ve zkoumané oblasti rezervace o rozloze 100 $km^2$ bude alespoň 1 stádo ale nejvýše 7 stád? 

\begin{lstlisting}[language=R, showstringspaces=false, basicstyle=\small]
    # P(1<=X<=7) = P(X <= 7) - P(X <= 0)
    # V oblasti 100km^2 bude alespon 1 ale nejvyse 7 stad
    ppois(7, lt) - ppois(0, lt) # s pravdepodobnosti 93,06 %
\end{lstlisting}

\subsubsection{}
c) Jaká je pravděpodobnost, že ve zkoumané oblasti rezervace o rozloze 100 $km^2$ bude alespoň 5 stád, jestliže už 2 stáda v ní byla nalezena?

\begin{lstlisting}[language=R, showstringspaces=false, basicstyle=\small]
    # V oblasti 100km^2 bude alespon 5 stad, jestlize 2 stada byla nalezena
    (1-ppois(4,lt)) / (1-ppois(1,lt)) # 0.4085801
\end{lstlisting}

\subsubsection{}
d) Redakce přírodovědeckého časopisu chce provést focení stáda. Za tímto účelem vybere 10 oblastí a do každé oblasti umístí sledovací zařízení, přičemž každé zařízení zvládne monitorovat oblast o rozloze 5 $km^2$. Aby focení bylo úspěšné, musí v alespoň jedné monitorované oblasti zachytit alespoň jedno stádo (přepokládáme, že zařízení jsou spolehlivá, takže pokud se stádo v oblasti vyskytne, pak pořízené fotky jsou použitelné). Jaká je pravděpodobnost, že získají fotky pro blížící se vydání časopisu?

\begin{lstlisting}[language=R, showstringspaces=false, basicstyle=\small]
    lambda = 1/25 
    t = 50 
    # P(X >= 1) = P(X > 0) = 1 - P(X <= 0)
    # V oblasti 50km^2 bude alespon jedno stado
    1 - ppois(0, lambda*t) # s pravdepodobnosti 86,47 %
\end{lstlisting}


\newpage
\subsection{}
Učitel opraví v průměru 6 domácích úkolů za hodinu.

\subsubsection{}
a) Jaká je pravděpodobnost, že za osmihodinovou pracovní dobu stihne opravit více než 50 úkolů?

\begin{lstlisting}[language=R, showstringspaces=false, basicstyle=\small]
    # X... pocet opravenych domacich ukolu za 1h
    # X - pouzijeme Poissonovu vetu
    lambda = 6 # cetnost vyskytu za hodinu
    t = 8 # behem 8mi hodin
    lt = lambda*t # parametr Poissonova rozdeleni
    
    # P(x > 50) = 1 - P(X <= 49)
    # Opravi vice jak 50 ukolu
    1 - ppois(50, lt) # s 35,13% pravdepodobnosti    
\end{lstlisting}

\subsubsection{}
b) Jaká je pravděpodobnost, že za osmihodinovou pracovní dobu stihne opravit alespoň 45 ale nejvýše 55 úkolů?

\begin{lstlisting}[language=R, showstringspaces=false, basicstyle=\small]
    # Opravi 45 az 55 ukolu
    ppois(55, lt) - ppois(45-1, lt) # s 54,67% sanci
\end{lstlisting}

\subsubsection{}
c) Jaká je pravděpodobnost, že za osmihodinovou pracovní dobu stihne opravit alespoň 55 úkolů, jestliže už 40 úkolů opravil?

\begin{lstlisting}[language=R, showstringspaces=false, basicstyle=\small]
    # Opravi 55 ukolu jestlize 40 opravil
    (1 - ppois(54, lt)) / (1 - ppois(39, lt)) # s pravdepodobnosti 19,4 %
\end{lstlisting}

\subsubsection{}
d) Mezi úkoly je vždy 20 \% kompletně špatně vypracovaných úkolů. Učitel opravuje úkoly tak dlouho, dokud postupně nenarazí na 3 takové špatné úkoly, pak si dá čokoládový bonbon na nervy. Určete pravděpodobnost, že učitel bude muset opravit více než 17 úkolů, než si dá bonbon.

\begin{lstlisting}[language=R, showstringspaces=false, basicstyle=\small]
    # X ... pocet pokusu nez vybereme spatny ukol
    # X ~ NB(k = 3,p = 0.2)
    x = 17
    k = 3
    p = 0.2
    # P(X > 15) = 1 - P(X <= 15)
    # Narazi na tri spatne ukoly nejdrive za 17 pokusu
    1-pnbinom(x - k, k, p) # s pravdepodobnosti 30,96 %
\end{lstlisting}


\newpage
\subsection{}
Doba, po kterou server funguje bez problémů, aniž by potřeboval restartovat, má exponenciální rozdělení. Průměrná doba do nutného restartu je 1 rok.

\subsubsection{}
a) Načrtněte hustotu pravděpodobnosti uvedené náhodné veličiny a její distribuční funkci.

\begin{lstlisting}[language=R, showstringspaces=false, basicstyle=\small]
    # X ... doba do restartu ve dnech
    ## X ~ Exp(lambda), kde E(X)=1/lambda
    lambda = 1/365
    
    x = seq(from = 0, to = 2500, by = 1)
    f_x = dexp(x-1, lambda)
    plot(x, f_x, cex = 0.1, main="Hustota pravdepodobnosti po dnech")
    grid()
    
    F_x = pexp(x-1, lambda)
    plot(x, F_x, type='s', main="Distribucni funkce po dnech")
    grid()  
\end{lstlisting}

\subsubsection{}
b) Jaká je pravděpodobnost, že server bude potřebovat restartovat nejdříve za 13 měsíců? Výsledek zaznačte do náčrtku hustoty pravděpodobnosti z bodu a). 

\begin{lstlisting}[language=R, showstringspaces=false, basicstyle=\small]
    # Restart bude potreba nejdrive za 13 mesicu
    1 - pexp(365/12*13, lambda) # s pravdepodobnosti 33,85%
    x = seq(from = 0, to = 2500, by = 1)
    f_x = dexp(x-1, lambda)
    plot(x, f_x, cex = 0.1, main="Hustota pravdepodobnosti a restart po 13 mesicich")
    lines(c(0, 2500),c(dexp(365/12*13, lambda), dexp(365/12*13, lambda)))
    lines(c(365/12*13, 365/12*13),c(0, max(f_x)))
    grid()
\end{lstlisting}

\subsubsection{}
c) Server jede už 12 měsíců bez restartu, jaká je pravděpodobnost, že bude potřeba jej restartovat v následujících 14 dnech?

\begin{lstlisting}[language=R, showstringspaces=false, basicstyle=\small]
    # Pravdepodobnost restartu po roce pouzivani v nasledujicich 14 dnech
    pexp(365 + 365/52*2, lambda) - pexp(365, lambda) # je 1,38 %.
\end{lstlisting}

\subsubsection{}
d) Do jaké doby bude nutné provést restart serveru s 90 \% pravděpodobností?

\begin{lstlisting}[language=R, showstringspaces=false, basicstyle=\small]
    # P(X<t)=0,9 -> F(t)=0,9 -> t... 90% kvantil
    # Server bude treba restartovat s 90 % pravdepodobnosti
    qexp(0.9, lambda) # po 840,4 dnech.    
\end{lstlisting}


\newpage
\subsection{}
Systolický krevní tlak dospělých má normální rozdělení se střední hodnotou 112 mmHg a směrodatnou odchylkou 10 mmHg.

\subsubsection{}
a) Načrtněte hustotu pravděpodobnosti uvedené náhodné veličiny a její distribuční funkci.

\begin{lstlisting}[language=R, showstringspaces=false, basicstyle=\small]
    # Systolicky krevni tlak, normalni rozdeleni se stredni hodnotou 112 
    # a smerodatnou odchylkou 10
    mu = 112
    sigma = 10
    
    # vykreslime si Hustotu pravdepodobnosti
    x = seq(from = 70, to = 150, by = 0.1)
    f_x = dnorm(x, mean=mu, sd=sigma)
    plot(x, f_x, cex = 0.01, main="Hustota pravdepodobnosti")
    grid()
    
    # vykreslime si Distribucni funkci
    F_x = pnorm(x, mean=mu, sd=sigma)
    plot(x, F_x, type = 'l', main="Distribucni funkce")
    grid()
\end{lstlisting}

\subsubsection{}
b) Kolik procent dospělých má systolický krevní tlak mimo ideální rozmezí, které je 90 až 120 mmHg? Výsledek zaznačte do náčrtku hustoty pravděpodobnosti z bodu a).

\begin{lstlisting}[language=R, showstringspaces=false, basicstyle=\small]
    # Tlak mimo rozmezi 90 az 120g
    1- (pnorm(120, mean=mu, sd=sigma) - pnorm(90, mean=mu, sd=sigma)) 
    # s pravdepodobnosti 22,58 %
    
    # Zaznaceni do grafu
    # x = c(seq(from = 50, to = 90, by = 0.1),seq(from = 120, to = 180, by = 0.1))
    # Vysrafovana oblast je pod krivkou v levo od hodnoty 90 
    # a v pravo od hodnoty x=120 a y<0
    x = seq(from = 70, to = 150, by = 0.1)
    f_x = dnorm(x, mean=mu, sd=sigma)
    plot(x, f_x, cex = 0.01, main="Hustota pravdepodobnosti")
    lines(c(90,90),c(0,max(f_x)))
    lines(c(120,120),c(0,max(f_x)))
    grid()
\end{lstlisting}

\subsubsection{}
c) Kolik procent dospělých má systolický krevní tlak nižší než 105 mmHg? Výsledek zaznačte do náčrtku distribuční funkce z bodu a).

\begin{lstlisting}[language=R, showstringspaces=false, basicstyle=\small]
    # Tlak nizsi nez 105mmHg
    pnorm(105, mean=mu, sd=sigma) # s pravdepodobnosti 24,2 %
    F_x = pnorm(x, mean=mu, sd=sigma)
    plot(x, F_x, type = 'l', main="Distribucni funkce s tlakem")
    lines(c(105,105),c(0,max(F_x)))
    lines(c(70,150),c(pnorm(105, mean=mu, sd=sigma),pnorm(105, mean=mu, sd=sigma)))
    grid()    
\end{lstlisting}

\subsubsection{}
d) Určete hodnotu 82. percentilu uvedené náhodné veličiny. Slovně ji interpretujte.

\begin{lstlisting}[language=R, showstringspaces=false, basicstyle=\small]
    # 82. percentil
    qnorm(0.82, mean=mu, sd=sigma) # je 121.15 mmHg
    # Vysvetleni percentilu: 82% lidi ma hodnotu systolickeho 
    # krevniho tlaku do hranice 121.15 mmHg
\end{lstlisting}


\end{document}